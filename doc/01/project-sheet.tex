\documentclass[8pt]{article}

\usepackage[utf8]{inputenc}
\usepackage[ngerman]{babel}
\usepackage{amsmath}
\usepackage{nicefrac}
\usepackage{color}
\usepackage[hidelinks]{hyperref}

\usepackage[a4paper,top=2cm,left=4cm,right=4cm,bottom=2cm,marginparwidth=5cm]{geometry}
\usepackage{marginnote}
\usepackage[automark]{scrpage2}
\pagestyle{scrheadings}

\newcommand{\fatnote}[1]{\marginnote{\textbf{#1}}}
\newcommand{\leftnote}[1]{\reversemarginpar\fatnote{#1}}
\newcommand{\rightnote}[1]{\normalmarginpar\fatnote{#1}}

\newcommand{\defin}[1]{\noindent #1\vspace{0.3cm}}
\newcommand{\ldefin}[2]{\leftnote{#1}\defin{#2}}
\newcommand{\rdefin}[2]{\rightnote{#1}\defin{#2}}

\newcommand{\todo}[1]{\textcolor{red}{\textbf{TODO}: #1}}

\newcommand{\coursename}{\@empty}
\newcommand{\groupno}{\@empty}

\newcommand{\course}[2]{\renewcommand{\coursename}{#1}\renewcommand{\groupno}{#2}}
\newcommand{\beginsheet}{\clearscrheadfoot\ihead[]{Kurs: \coursename}\ohead[]{Gruppe \groupno}\ofoot[]{\pagemark}\ifoot[]{Dieses Dokument ist Teil der Dokumentation}}

\newcommand{\email}[1]{\href{mailto:#1}{#1}}


%%%%%%%%%%%%%%%%%%%%%%%%%%%%%%%%%%%%%%%%%%%%%%%%%%%%%%%%%%%%%%%%%%%%%%%%%%%%%%%%%%%%%%
%% Oberhalb dieses Blocks nichts ändern
%%%%%%%%%%%%%%%%%%%%%%%%%%%%%%%%%%%%%%%%%%%%%%%%%%%%%%%%%%%%%%%%%%%%%%%%%%%%%%%%%%%%%%

\setlength{\parskip}{1.0em}
\setlength{\parindent}{0pt}

%% TODO: Hier die fehlende Gruppennummer einfügen
\course{Lisp Kurs -- Roboterprogrammierung in Lisp}{0}

\begin{document}

\beginsheet

\section*{Projekt-Übersicht}
Stand: \today\\[0.25cm]
Mitglieder:\\[0.25cm]
%% TODO: Hier die Mitglieder samt E-Mail-Adressen eintragen
\begin{tabular}{|p{0.5\columnwidth}|p{0.5\columnwidth}|}
  \hline
  \textbf{Name} & \textbf{E-Mail} \\
  \hline
  \hline
  Tom Gehrke & \email{tgehrke@tzi.de} \\
  \hline
  Hannes Welling & \email{h.welling@tzi.de} \\
  \hline
  Andreas Romero Früh & \email{aromero@tzi.de} \\
  \hline
\end{tabular}
\vspace{0.25cm}\\
Betreuer:\\[0.25cm]
\begin{tabular}{|p{0.5\columnwidth}|p{0.5\columnwidth}|}
  \hline
  \textbf{Name} & \textbf{E-Mail} \\
  \hline
  \hline
  Jan Winkler & \email{jwinkler@uni-bremen.de} \\
  \hline
\end{tabular}
\vspace{0.25cm}\\
Projekt-Repository: \url{git@github.com:phyrog/fetch_nxt.git}

\subsection*{Thema und Problemstellung}
Unser Projekt trägt den Codenamen \textit{SFMB} (\texttt{\$sudo fetch my ball}).

Unsere gewählte Problemstellung beinhaltet das Finden, Aufheben und Zurückbringen eines farbigen Balls in einem bekannten rechteckigen Bereich. Der Bereich ist durch einen einfarbigen Rand und gegebenenfalls eine Bande begrenzt. Größe des Bereichs und Dimensionen sowie Farbe des Randes und der Bande sind im Verlauf des Projekts festzulegen. Der Ball soll an den Startpunkt des Roboters transportiert werden.

\subsection*{Angestrebte Problemlösung}
Zur Lösung des Problems werden verschiedene Sensoren der NXT-Plattform verwenden. Zum Finden des Balls soll der Ultraschallsensor verwendet werden. Ein Farbsensor soll unseren Roboter erkennen lassen, sobald er den farbigen Rand überfährt, um ein Verlassen des markierten Bereichs zu verhindern.

\end{document}
